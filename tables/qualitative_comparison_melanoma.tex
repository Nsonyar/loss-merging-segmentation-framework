% Please add the following required packages to your document preamble:
% \usepackage{graphicx}
% \usepackage[table,xcdraw]{xcolor}
% If you use beamer only pass "xcolor=table" option, i.e. \documentclass[xcolor=table]{beamer}
\begin{table}[H]
  \centering
  \resizebox{\textwidth}{!} &
    {\color[HTML]{FFFFFF} Strategy} &
    {\color[HTML]{FFFFFF} IoU} &
    {\color[HTML]{FFFFFF} IoU 0} &
    {\color[HTML]{FFFFFF} IoU 1} &
    {\color[HTML]{FFFFFF} PPV} &
    {\color[HTML]{FFFFFF} TPR} &
    {\color[HTML]{FFFFFF} PPV vs. TPR} \\ \hline
  1 &
    158 &
    Focal &
    0.64 &
     &
    0.659 &
    0.821 &
    0.497 &
    0.832 &
    0.793 &
    PPV \\ \hline
  2 &
    185 &
    CE,TL &
    0.64 &
    NWS &
    0.671 &
    0.812 &
    0.529 &
    0.822 &
    0.814 &
    PPV \\ \hline
  \end{tabular}%
  }
  \caption[Quantitative results for qualitative analysis (Skin Lesion)]{The table presents two distinct models that will be used in this section for a qualitative comparison. Model no.1 is constructed using a baseline setup, while model no.2 has been trained with the proposed loss merging framework. The $IoU$ column reflects the average outcomes contrasting the foreground class against the background class where the columns $IoU_0,IoU_1$ represent the Background and Foreground class individually. The column titled \squote{PPV vs. TPR} illustrates the trade-off between a high \acf{PPV} and low \acf{TPR}, or vice versa, for each model.}
  \label{tab:qualitative_comparison_melanoma}
  \end{table}