% Please add the following required packages to your document preamble:
% \usepackage{graphicx}
% \usepackage[table,xcdraw]{xcolor}
% If you use beamer only pass "xcolor=table" option, i.e. \documentclass[xcolor=table]{beamer}
\begin{table}[H]
    \centering
    \resizebox{\textwidth}{!} &
      {\color[HTML]{FFFFFF} Strategy} &
      {\color[HTML]{FFFFFF} IoU} &
      {\color[HTML]{FFFFFF} IoU 0} &
      {\color[HTML]{FFFFFF} IoU 1} &
      {\color[HTML]{FFFFFF} IoU 2} &
      {\color[HTML]{FFFFFF} IoU 3} &
      {\color[HTML]{FFFFFF} PPV} &
      {\color[HTML]{FFFFFF} TPR} &
      {\color[HTML]{FFFFFF} PPV vs. TPR} \\ \hline
    1 &
      73 &
      CE &
      1.00 &
      - &
      0.697 &
      0.974 &
      0.756 &
      0.471 &
      0.586 &
      0.807 &
      0.832 &
      TPR \\ \hline
    2 &
      138 &
      HD,FL &
      1.00 &
      AVG &
      0.762 &
      0.984 &
      0.751 &
      0.607 &
      0.705 &
      0.874 &
      0.834 &
      PPV \\ \hline
    \end{tabular}%
    }
    \caption[Quantitative results for qualitative analysis (Medaka)]{The table presents two distinct models that will be used in this section for a qualitative comparison. Model no.1 is constructed using a baseline setup, while model no.2 has been trained with the proposed loss merging framework. The columns $IoU_0,\hdots,IoU_3$ represent the four classes, Background, Bulbus, Atrium and Ventricle.  The column titled \squote{PPV vs. TPR} illustrates the trade-off between a high \acf{PPV} and low \acf{TPR}, or vice versa, for each model.}
    \label{tab:qualitative_comparison_medaka}
    \end{table}