\chapter{Metric Comparison}
\label{chap:metric_comparison}
It can be challenging to determine the best metric, as users may prioritize different aspects of the classification results depending on the task. Semantic segmentation often involves regions of interest much smaller than the background region. When valid for the entire dataset, the data is considered unbalanced. Therefore, choosing the right metric that accurately reflects changes in unstable situations is crucial.

Table \ref{tab:metric_comparison_1} compares a situation where one pixel is misclassified as a false negative for both balanced and unbalanced labels. Table \ref{tab:metric_comparison_2} compares a similar situation focusing on changing a false positive misclassified pixel. Both tables show that the \ac{IoU} and \ac{DSC} metrics reflect the performance change as expected for balanced and imbalanced data. In contrast, \ac{Acc} does not change as it includes true negatives in its calculation, providing a very high misleading score for both cases.

% Please add the following required packages to your document preamble:
% \usepackage{graphicx}
% \usepackage[table,xcdraw]{xcolor}
% If you use beamer only pass "xcolor=table" option, i.e. \documentclass[xcolor=table]{beamer}
\begin{table}[H]
    \centering
    \resizebox{\textwidth}{!}{%
    \begin{tabular}{llllllllllll}
     &
      \multicolumn{7}{c}{False negative misclassification} &
       &
       &
       &
       \\ \cline{2-12} 
    \multicolumn{1}{c|}{\textbf{Balanced}} &
      \multicolumn{1}{c|}{\cellcolor[HTML]{000000}{\color[HTML]{FFFFFF} PPV}} &
      \multicolumn{1}{c|}{\cellcolor[HTML]{000000}{\color[HTML]{FFFFFF} TPR}} &
      \multicolumn{1}{c|}{\cellcolor[HTML]{000000}{\color[HTML]{FFFFFF} TNR}} &
      \multicolumn{1}{c|}{\cellcolor[HTML]{000000}{\color[HTML]{FFFFFF} NPV}} &
      \multicolumn{1}{c|}{\cellcolor[HTML]{000000}{\color[HTML]{FFFFFF} IoU}} &
      \multicolumn{1}{c|}{\cellcolor[HTML]{000000}{\color[HTML]{FFFFFF} Dice}} &
      \multicolumn{1}{c|}{\cellcolor[HTML]{000000}{\color[HTML]{FFFFFF} ACC}} &
      \multicolumn{1}{c|}{\cellcolor[HTML]{00A9CE}{\color[HTML]{FFFFFF} TP}} &
      \multicolumn{1}{c|}{\cellcolor[HTML]{99DDFD}{\color[HTML]{FFFFFF} FP}} &
      \multicolumn{1}{c|}{\cellcolor[HTML]{6638B6}{\color[HTML]{FFFFFF} FN}} &
      \multicolumn{1}{c|}{\cellcolor[HTML]{9B7DD4}{\color[HTML]{FFFFFF} TN}} \\ \hline
    \multicolumn{1}{|c|}{\cellcolor[HTML]{000000}{\color[HTML]{FFFFFF} TP(11) : TN(13)}} &
      \multicolumn{1}{c|}{1.000} &
      \multicolumn{1}{c|}{0.917} &
      \multicolumn{1}{c|}{1.000} &
      \multicolumn{1}{c|}{0.929} &
      \multicolumn{1}{c|}{0.917} &
      \multicolumn{1}{c|}{0.957} &
      \multicolumn{1}{c|}{0.960} &
      \multicolumn{1}{c|}{11} &
      \multicolumn{1}{c|}{0} &
      \multicolumn{1}{c|}{1} &
      \multicolumn{1}{c|}{13} \\ \hline
    \multicolumn{1}{|c|}{\cellcolor[HTML]{000000}{\color[HTML]{FFFFFF} TP(12) : TN(13)}} &
      \multicolumn{1}{c|}{1.000} &
      \multicolumn{1}{c|}{1.000} &
      \multicolumn{1}{c|}{1.000} &
      \multicolumn{1}{c|}{1.000} &
      \multicolumn{1}{c|}{1.000} &
      \multicolumn{1}{c|}{1.000} &
      \multicolumn{1}{c|}{1.000} &
      \multicolumn{1}{c|}{12} &
      \multicolumn{1}{c|}{0} &
      \multicolumn{1}{c|}{0} &
      \multicolumn{1}{c|}{13} \\ \hline
    \multicolumn{1}{|c|}{\cellcolor[HTML]{000000}{\color[HTML]{FFFFFF} Difference}} &
      \multicolumn{1}{c|}{0.00\%} &
      \multicolumn{1}{c|}{8.33\%} &
      \multicolumn{1}{c|}{0.00\%} &
      \multicolumn{1}{c|}{7.14\%} &
      \multicolumn{1}{c|}{8.33\%} &
      \multicolumn{1}{c|}{4.35\%} &
      \multicolumn{1}{c|}{4.00\%} &
      \multicolumn{1}{c}{} &
      \multicolumn{1}{c}{} &
      \multicolumn{1}{c}{} &
      \multicolumn{1}{c}{} \\ \cline{1-8}
     &
       &
       &
       &
       &
       &
       &
       &
       &
       &
       &
       \\ \cline{2-12} 
    \multicolumn{1}{l|}{\textbf{Imbalanced}} &
      \multicolumn{1}{l|}{\cellcolor[HTML]{000000}{\color[HTML]{FFFFFF} PPV}} &
      \multicolumn{1}{l|}{\cellcolor[HTML]{000000}{\color[HTML]{FFFFFF} TPR}} &
      \multicolumn{1}{l|}{\cellcolor[HTML]{000000}{\color[HTML]{FFFFFF} TNR}} &
      \multicolumn{1}{l|}{\cellcolor[HTML]{000000}{\color[HTML]{FFFFFF} NPV}} &
      \multicolumn{1}{l|}{\cellcolor[HTML]{000000}{\color[HTML]{FFFFFF} IoU}} &
      \multicolumn{1}{l|}{\cellcolor[HTML]{000000}{\color[HTML]{FFFFFF} Dice}} &
      \multicolumn{1}{l|}{\cellcolor[HTML]{000000}{\color[HTML]{FFFFFF} ACC}} &
      \multicolumn{1}{l|}{\cellcolor[HTML]{00A9CE}{\color[HTML]{FFFFFF} TP}} &
      \multicolumn{1}{l|}{\cellcolor[HTML]{99DDFD}{\color[HTML]{FFFFFF} FP}} &
      \multicolumn{1}{l|}{\cellcolor[HTML]{6638B6}{\color[HTML]{FFFFFF} FN}} &
      \multicolumn{1}{l|}{\cellcolor[HTML]{9B7DD4}{\color[HTML]{FFFFFF} TN}} \\ \hline
    \multicolumn{1}{|l|}{\cellcolor[HTML]{000000}{\color[HTML]{FFFFFF} TP(2) : TN(22)}} &
      \multicolumn{1}{l|}{1.000} &
      \multicolumn{1}{l|}{0.667} &
      \multicolumn{1}{l|}{1.000} &
      \multicolumn{1}{l|}{\cellcolor[HTML]{FFFFFF}0.957} &
      \multicolumn{1}{l|}{0.667} &
      \multicolumn{1}{l|}{0.800} &
      \multicolumn{1}{l|}{0.960} &
      \multicolumn{1}{l|}{2} &
      \multicolumn{1}{l|}{0} &
      \multicolumn{1}{l|}{1} &
      \multicolumn{1}{l|}{22} \\ \hline
    \multicolumn{1}{|l|}{\cellcolor[HTML]{000000}{\color[HTML]{FFFFFF} TP(3) : TN(22)}} &
      \multicolumn{1}{l|}{1.000} &
      \multicolumn{1}{l|}{1.000} &
      \multicolumn{1}{l|}{1.000} &
      \multicolumn{1}{l|}{1.000} &
      \multicolumn{1}{l|}{1.000} &
      \multicolumn{1}{l|}{1.000} &
      \multicolumn{1}{l|}{1.000} &
      \multicolumn{1}{l|}{3} &
      \multicolumn{1}{l|}{0} &
      \multicolumn{1}{l|}{0} &
      \multicolumn{1}{l|}{22} \\ \hline
    \multicolumn{1}{|l|}{\cellcolor[HTML]{000000}{\color[HTML]{FFFFFF} Difference}} &
      \multicolumn{1}{l|}{0.00\%} &
      \multicolumn{1}{l|}{33.33\%} &
      \multicolumn{1}{l|}{0.00\%} &
      \multicolumn{1}{l|}{4.35\%} &
      \multicolumn{1}{l|}{33.33\%} &
      \multicolumn{1}{l|}{20.00\%} &
      \multicolumn{1}{l|}{4.00\%} &
       &
       &
       &
       \\ \cline{1-8}
    \end{tabular}%
    }
    \caption[Metric comparison - False negative misclassification]{Tables illustrating the score difference in percent by misclassifying one pixel as false negative for a balanced region at the top, and an imbalanced situation at the bottom. We can see that the \ac{IoU} and \ac{DSC} properly reflect the change by a much larger difference percentage in comparison to accuracy. Accuracy does not change as it includes true negatives in its calculation providing in both cases a very high misleading score. As the misclassified pixel was a false negative, precision stays unaffected while recall decreases by $33.33\%$}
    \label{tab:metric_comparison_1}
    \end{table}
% Please add the following required packages to your document preamble:
% \usepackage{graphicx}
% \usepackage[table,xcdraw]{xcolor}
% If you use beamer only pass "xcolor=table" option, i.e. \documentclass[xcolor=table]{beamer}
\begin{table}[H]
    \centering
    \resizebox{\textwidth}{!}{%
    \begin{tabular}{llllllllllll}
     &
      \multicolumn{7}{c}{False positive misclassification} &
       &
       &
       &
       \\ \cline{2-12} 
    \multicolumn{1}{l|}{\textbf{Balanced}} &
      \multicolumn{1}{l|}{\cellcolor[HTML]{000000}{\color[HTML]{FFFFFF} PPV}} &
      \multicolumn{1}{l|}{\cellcolor[HTML]{000000}{\color[HTML]{FFFFFF} TPR}} &
      \multicolumn{1}{l|}{\cellcolor[HTML]{000000}{\color[HTML]{FFFFFF} TNR}} &
      \multicolumn{1}{l|}{\cellcolor[HTML]{000000}{\color[HTML]{FFFFFF} NPV}} &
      \multicolumn{1}{l|}{\cellcolor[HTML]{000000}{\color[HTML]{FFFFFF} IoU}} &
      \multicolumn{1}{l|}{\cellcolor[HTML]{000000}{\color[HTML]{FFFFFF} Dice}} &
      \multicolumn{1}{l|}{\cellcolor[HTML]{000000}{\color[HTML]{FFFFFF} ACC}} &
      \multicolumn{1}{l|}{\cellcolor[HTML]{00A9CE}{\color[HTML]{FFFFFF} TP}} &
      \multicolumn{1}{l|}{\cellcolor[HTML]{99DDFD}{\color[HTML]{FFFFFF} FP}} &
      \multicolumn{1}{l|}{\cellcolor[HTML]{6638B6}{\color[HTML]{FFFFFF} FN}} &
      \multicolumn{1}{l|}{\cellcolor[HTML]{9B7DD4}{\color[HTML]{FFFFFF} TN}} \\ \hline
    \multicolumn{1}{|l|}{\cellcolor[HTML]{000000}{\color[HTML]{FFFFFF} TP(12) : TN(12)}} &
      \multicolumn{1}{l|}{0.923} &
      \multicolumn{1}{l|}{1.000} &
      \multicolumn{1}{l|}{0.923} &
      \multicolumn{1}{l|}{1.000} &
      \multicolumn{1}{l|}{0.923} &
      \multicolumn{1}{l|}{0.960} &
      \multicolumn{1}{l|}{0.960} &
      \multicolumn{1}{l|}{12} &
      \multicolumn{1}{l|}{1} &
      \multicolumn{1}{l|}{0} &
      \multicolumn{1}{l|}{12} \\ \hline
    \multicolumn{1}{|l|}{\cellcolor[HTML]{000000}{\color[HTML]{FFFFFF} TP(12) : TN(13)}} &
      \multicolumn{1}{l|}{1.000} &
      \multicolumn{1}{l|}{1.000} &
      \multicolumn{1}{l|}{1.000} &
      \multicolumn{1}{l|}{1.000} &
      \multicolumn{1}{l|}{1.000} &
      \multicolumn{1}{l|}{1.000} &
      \multicolumn{1}{l|}{1.000} &
      \multicolumn{1}{l|}{12} &
      \multicolumn{1}{l|}{0} &
      \multicolumn{1}{l|}{0} &
      \multicolumn{1}{l|}{13} \\ \hline
    \multicolumn{1}{|l|}{\cellcolor[HTML]{000000}{\color[HTML]{FFFFFF} Difference}} &
      \multicolumn{1}{l|}{7.69\%} &
      \multicolumn{1}{l|}{0.00\%} &
      \multicolumn{1}{l|}{7.69\%} &
      \multicolumn{1}{l|}{0.00\%} &
      \multicolumn{1}{l|}{7.69\%} &
      \multicolumn{1}{l|}{4.00\%} &
      \multicolumn{1}{l|}{4.00\%} &
       &
       &
       &
       \\ \cline{1-8}
     &
       &
       &
       &
       &
       &
       &
       &
       &
       &
       &
       \\ \cline{2-12} 
    \multicolumn{1}{l|}{\textbf{Imbalanced}} &
      \multicolumn{1}{l|}{\cellcolor[HTML]{000000}{\color[HTML]{FFFFFF} PPV}} &
      \multicolumn{1}{l|}{\cellcolor[HTML]{000000}{\color[HTML]{FFFFFF} TPR}} &
      \multicolumn{1}{l|}{\cellcolor[HTML]{000000}{\color[HTML]{FFFFFF} TNR}} &
      \multicolumn{1}{l|}{\cellcolor[HTML]{000000}{\color[HTML]{FFFFFF} NPV}} &
      \multicolumn{1}{l|}{\cellcolor[HTML]{000000}{\color[HTML]{FFFFFF} IoU}} &
      \multicolumn{1}{l|}{\cellcolor[HTML]{000000}{\color[HTML]{FFFFFF} Dice}} &
      \multicolumn{1}{l|}{\cellcolor[HTML]{000000}{\color[HTML]{FFFFFF} ACC}} &
      \multicolumn{1}{l|}{\cellcolor[HTML]{00A9CE}{\color[HTML]{FFFFFF} TP}} &
      \multicolumn{1}{l|}{\cellcolor[HTML]{99DDFD}{\color[HTML]{FFFFFF} FP}} &
      \multicolumn{1}{l|}{\cellcolor[HTML]{6638B6}{\color[HTML]{FFFFFF} FN}} &
      \multicolumn{1}{l|}{\cellcolor[HTML]{9B7DD4}{\color[HTML]{FFFFFF} TN}} \\ \hline
    \multicolumn{1}{|l|}{\cellcolor[HTML]{000000}{\color[HTML]{FFFFFF} TP(3) : TN(21)}} &
      \multicolumn{1}{l|}{0.750} &
      \multicolumn{1}{l|}{1.000} &
      \multicolumn{1}{l|}{0.955} &
      \multicolumn{1}{l|}{\cellcolor[HTML]{FFFFFF}1.000} &
      \multicolumn{1}{l|}{0.750} &
      \multicolumn{1}{l|}{0.857} &
      \multicolumn{1}{l|}{0.960} &
      \multicolumn{1}{l|}{3} &
      \multicolumn{1}{l|}{1} &
      \multicolumn{1}{l|}{0} &
      \multicolumn{1}{l|}{21} \\ \hline
    \multicolumn{1}{|l|}{\cellcolor[HTML]{000000}{\color[HTML]{FFFFFF} TP(3) : TN(22)}} &
      \multicolumn{1}{l|}{1.000} &
      \multicolumn{1}{l|}{1.000} &
      \multicolumn{1}{l|}{1.000} &
      \multicolumn{1}{l|}{1.000} &
      \multicolumn{1}{l|}{1.000} &
      \multicolumn{1}{l|}{1.000} &
      \multicolumn{1}{l|}{1.000} &
      \multicolumn{1}{l|}{3} &
      \multicolumn{1}{l|}{0} &
      \multicolumn{1}{l|}{0} &
      \multicolumn{1}{l|}{22} \\ \hline
    \multicolumn{1}{|l|}{\cellcolor[HTML]{000000}{\color[HTML]{FFFFFF} Difference}} &
      \multicolumn{1}{l|}{25.00\%} &
      \multicolumn{1}{l|}{0.00\%} &
      \multicolumn{1}{l|}{4.55\%} &
      \multicolumn{1}{l|}{0.00\%} &
      \multicolumn{1}{l|}{25.00\%} &
      \multicolumn{1}{l|}{14.29\%} &
      \multicolumn{1}{l|}{4.00\%} &
       &
       &
       &
       \\ \cline{1-8}
    \end{tabular}%
    }
    \caption[Metric comparison - False positive misclassification]{Similar as the tables above do these tables illustrate the score difference in percent by misclassifying one pixel as false positive for a balanced label at the top, and an imbalanced label at the bottom. We can see again that the \ac{IoU} and \ac{DSC} properly reflect the change by a much larger difference percentage in comparison to accuracy. As the misclassified pixel was a true positive, recall here stays unaffected while precision decreases by $25.00\%$}
    \label{tab:metric_comparison_2}
    \end{table}